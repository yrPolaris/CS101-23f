\titledquestion{SmallSum}
Given an array \(\langle a_1,\cdots,a_n\rangle\), Let
$$
    f_i=\sum_{j=1}^{i-1} a_j\mathbb{I}(a_j<a_i)
$$
Then, we define SmallSum:
$$
    SmallSum = \sum_{i = 1}^{n} f_i
$$

For example, for array \(\langle1,3,4,2,5\rangle\):
\begin{itemize}
    \item for element $1$: $f_1=0$
    \item for element $3$: $f_2=1$
    \item for element $4$: $f_3=1+3=4$
    \item for element $2$: $f_4=1$
    \item for element $5$: $f_5=1+3+4+2=10$
\end{itemize}
So the SmallSum is $0+1+4+1+10=16$.

Most of you can come up with the $\Theta(n^2)$ solution, but let's think it in another way.  
\begin{parts}
    \part[1] For the example above, for an element $a_k$, how many times does it add to the sum? 
%%%%%%%%%%%%%%%%%%%%%%%%%%%%%%%%%%%%%%%%%%%%%%%%%%%%%%%%%%%%%%%%%%
% On the correct choice, replace `\choice' with `\CorrectChoice'.
%%%%%%%%%%%%%%%%%%%%%%%%%%%%%%%%%%%%%%%%%%%%%%%%%%%%%%%%%%%%%%%%%%
    \begin{choices}
        \choice $\sum_{i=1}^{k} \mathbb{I}(a_k<a_i)$  
        \choice $\sum_{i=1}^{k} \mathbb{I}(a_k \geq a_i)$  
        \CorrectChoice $\sum_{i=k+1}^{n} \mathbb{I}(a_k<a_i)$
        \choice $\sum_{i=k+1}^{n} \mathbb{I}(a_k \geq a_i)$  
    \end{choices}


\part[8] Fill in the blanks in the code snippet below.
  (Hint: relate this algorithm to counting inversions.) \\ 
Consider alternating the Enhance Merge Sort algorithm to solve the problem. \\

The solution has 2 functions:

%%%%%%%%%%%%%%%%%%%%%%%%%%%%%%%%%%%%%%%%%%%%%%%%%%%%%%%%%%%%%
		% Replace the underline with your answer.
%%%%%%%%%%%%%%%%%%%%%%%%%%%%%%%%%%%%%%%%%%%%%%%%%%%%%%%%%%%%%

\begin{cpp}
int split(int arr[], int left, int right){
    if(left == right) return 0;
    int mid = (left + right)/2;
    int result = split(arr, left, mid) + split(arr, mid + 1, right);
    return result + merge(arr, left, mid, right);
}
\end{cpp}

The following function calculates the contribution of \(\langle a_{left},\cdots,a_{mid}\rangle\) to the sum.

\begin{cpp}
int merge(int arr[], int left, int mid, int right){
    int length = right - left + 1;
    int* temp = new int[length];
    int i = 0;
    int p1 = left;
    int p2 = mid + 1;
    int result = 0;
    while(p1 <= mid && p2 <= right){
        if(arr[p1]<arr[p2]){
            // Add arr[p1]'s contribution to final answer.
            result += (right-p2+1)*arr[p1];
            // move the elements of arr into temp.
            // (for the following three lines, you may not fill all of them.)
            temp[i] = arr[p1];
            i++;
            p1++;
        }
        else{
            // move the elements of arr into temp.
            // (for the following three lines, you may not fill all of them.)
            temp[i] = arr[p2];
            i++;
            p2++;
        }
    }
    while(p1 <= mid){
        temp[i++] = arr[p1++];
    }
    while(p2 <= right){
        temp[i++] = arr[p2++];
    }
    for (int j = 0; j < length ; j++){
        arr[left+j] = temp[j];
    }
    delete[] temp;
    return result;
}
\end{cpp}


  \part[3] What's the time complexity of our algorithm? Please answer in the form of \(\Theta(\cdot)\) and prove your answer.
  \begin{solution}
    %%%%%%%%%%%%%%%%%%%%%%%%%%%%%%%%%%%%%%%%%%%%%%%%%%%%%%%%%%%%%
		% Replace the `vspace{?cm}' with your answer.
	%%%%%%%%%%%%%%%%%%%%%%%%%%%%%%%%%%%%%%%%%%%%%%%%%%%%%%%%%%%%%
    $\theta(nlogn)$\\
    proof:
    For function merge, other operations are all $\theta(1)$, but\\
    For the three 'while', the first while's best case is $\theta(\frac{n}{2})$ and the worst case is $\theta(n)$ and other two 'while' is correspongding to the first 'while' by p1 and p2.
    Which means the total time complexity of three while is $\theta(n)$.\\
    For the 'for' loop, because length is n, then the time complexity is $\theta(n)$
    So the time complexity of the function merge is $\theta(n)$.\\
    
    For function split, it calls itself twice whose time complexity is half of it since it's equally splited.\\
    Besides it calls function merge too.\\
    So $T(n) = T(\frac{n}{2}) + \theta(n)$.Whose solution is nlog(n).\\
    So the time complexity of the algorithm is $\theta(n)$.
  \end{solution}
\end{parts}