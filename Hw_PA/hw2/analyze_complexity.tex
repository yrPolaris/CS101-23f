\titledquestion{Analysing the Time Complexity of a C++ Function}[4]

What is the time complexity of \lstinline{fun} (in the form of \(\Theta(f(n))\), where \(n\) is the size of the \lstinline{vector a})?

\begin{cpp}
void fun(std::vector<int> a) {
    int n = a.size();
    for (int i = 1; i < n; i *= 2) {
        // do O(1) operations
        for (int j = 0; j < n; j += i * 2) {
            // do O(1) operations
            for (int k = 0; k < i; ++k) {
                // do O(1) operations
            }
        }
    }
}
\end{cpp}

\textbf{NOTE}: Please clearly demonstrate your complexity analysis: you should give the complexity of the basic parts of an algorithm first, and then analyse the complexity of larger parts. The answer of the total complexity alone only accounts for 1pt.

\begin{solution}
    First, for the third loop, its time complexity is i for each the second loop's execution.\\
    Next,Because the second and third loop's time complexity are connected directly to variable i, we analyse their time complexity together. When the third loop is exactly executed by i times, the corresponded second loop will be executed by $\frac{n}{2i}$ times.\\
    Then, for the first loop, its time complexity is $log_{2}n$ for i increases by factor 2.\\    
    So the total time complexity is $log_{2}n * (i * \frac{n}{2i})$.\\
    Since constant doesn't have influence on the time complexity, the final $f(n)$ is\\
    $nlog(n)$. 
\end{solution}