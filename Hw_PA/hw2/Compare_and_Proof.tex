\titledquestion{Compare and Proof}

For each pair of functions \(f(n)\) and \(g(n)\), give your answer whether \(f(n) = o(g(n))\), \(f(n) = \omega(g(n))\) or \(f(n) = \Theta(g(n))\).  Give a \textbf{proof} of your answers.

\begin{parts}
    \part[3] \par
    \[f(n) = \log(n!)\]
    \[g(n) = \log(n^n)\]
    
    \begin{solution}
    According to 
    \begin{flalign}
        \begin{split}
            \frac{f(n)}{g(n)} &= \frac{log(n!)}{long(n^{n})}\\
                              &< \frac{nlogn}{nlogn} = 1\\
        \end{split}
    \end{flalign}
    While
    \begin{flalign}
        \begin{split}
            \lim\limits_{n \rightarrow \infty}\frac{f(n)}{g(n)} &> \lim\limits_{n \rightarrow \infty}\frac{0*2^{0} + 1*2^{1} + ... + (\lfloor log_{2}n \rfloor-1)*2^{\lfloor log_{2}n \rfloor-1} + \lfloor log_{2}n \rfloor*(n - 2^{\lfloor log_{2}n \rfloor} + 1)}{nlogn}\\
                                                                &> \lim\limits_{n \rightarrow \infty}\frac{(logn-2)2^{logn-2}}{nlogn}\\
                                                                &= \lim\limits_{n \rightarrow \infty}\frac{nlogn-2n}{4nlogn}\\
                                                                &= \frac{1}{4}\\
        \end{split}                             
    \end{flalign}
    So $f(n) = \Theta(g(n))$
    \end{solution}
\newpage
    \part[3] \par
    \[f(n) = n^{1+\varepsilon}, \qquad (\varepsilon\in \mathbb{R}^+)\]
    \[g(n) = n(\log n)^k, \qquad (k\in \mathbb{Z}^+)\]
    
    \begin{solution}
    Suppose $f(n) = o(g(n))$\\
    Set $\varepsilon = 2k$\\
    $\lim\limits_{n \rightarrow \infty}\frac{f(n)}{g(n)} = \lim\limits_{n \rightarrow \infty}(\frac{n^{2}}{logn})^{k} = \infty$\\
    Wrong\\

    As the same, $f(n) \neq \Theta (g(n))$\\
    Proof: $f(n) = \omega(g(n))$
    \begin{flalign}
        \begin{split}
            \lim\limits_{n \rightarrow \infty}\frac{f(n)}{g(n)} &= \lim\limits_{n \rightarrow \infty}(\frac{n^{\varepsilon}}{(logn)^{k}})\\
        \end{split}
    \end{flalign}
    For $\varepsilon \in \mathbb{R}^+$ and $k\in \mathbb{Z}^+$, $\lim\limits_{n \rightarrow \infty}(n^{\varepsilon}) = \infty$ $\lim\limits_{n \rightarrow \infty}((logn)^{k}) = \infty$.\\
    According to L'Hospital Theorem.\\
    $\lim\limits_{n \rightarrow \infty}\frac{f(n)}{g(n)}$ = $\lim\limits_{n \rightarrow \infty}\frac{n^{\varepsilon}(\varepsilon)}{kln(n)^{k-1}}(ln2)^{k}$\\
    For any i s.t. k-i > 0, there will be that $\lim\limits_{n \rightarrow \infty}[k*(k-1)*(k-2)...*(k-i+1)ln(n)^{k-i}] = \infty$ and $\lim\limits_{n \rightarrow \infty}(n^{\varepsilon}(\varepsilon)^{k-i}(ln2)^{k}) = \infty$. So it can use L'Hospital Theorem continously.\\
    While k-i = 0, there will be that $\lim\limits_{n \rightarrow \infty}\frac{f(n)}{g(n)} = \lim\limits_{n \rightarrow \infty}\frac{n^{\varepsilon}(\varepsilon)^{k}(ln2)^{k}}{k!}$.\\
    Because $(ln2)^{k}, (\varepsilon)^{k}$, and $ k!$ are constant.\\
    $\lim\limits_{n \rightarrow \infty}\frac{f(n)}{g(n)} = \lim\limits_{n \rightarrow \infty}\frac{n^{\varepsilon}(\varepsilon)^{k}(ln2)^{k}}{k!} = \infty$.\\
    i.e. $f(n) = \omega g(n)$.
    \end{solution}
\end{parts}
